\def \citeuci[#1]{\cite[l. #1]{uci}}
\section{UCI}\label{sec:uci}
\subsection{Allgemein}\label{subsec:allgemein}
UCI ist eine standardisierte Schnittstelle zwischen grafischer Oberfl\"ache und Schachengine.
Die Kommunikation findet \"uber die Standardein- und Ausgabe (ab sofort stdin und stdout) der beiden Programme ab.\citeuci[9]
\newline
Die Engine soll zu jedem Zeitpunkt Befehle entgegennehmen und verarbeiten k\"onnen.\citeuci[15]
Deshalb ist in Baguette ein eigener Thread, genannt UCI-Thread, f\"ur die Kommunikation zust\"andig.
\newline
Den Programmablauf kann man in zwei Phasen einteilen:
\begin{enumerate}
    \item Phase: Initialisierung
    \item Phase: Schachspielen
\end{enumerate}
\subsubsection{Phase 1 Initialisierung}
Diese Phase wird nicht im UCI-Thread sondern im Controller-Thread ausgef\"uhrt.
Zuerst best\"atigen sich Engine und GUI, dass UCI verwendet wird. \citeuci[59 - 66]
Dann meldet sich die Engine mit ihrem Namen und Authoren an.
Die Engine gibt der GUI mithilfe des \textit{option} Befehl alle m\"oglichen Optionen an.
\subsubsection{UCI}
Die UCI-Klasse verbindet die UCIBridge mit dem Controller.
Da zu jedem Zeitpunkt nur ein UCI-Thread existieren soll, ist die UCI-Klasse als threadsicherer Singleton implementiert.
Der UCI-Thread startet die awaitCommandsForever Methode der UCI-Klasse.

\subsubsection{UCIBridge}
UCIBridge bildet die Verbindung zur grafischen Oberfl\"ache.
Keine andere Klasse hat Zugriff auf stdin und stdout, da Java keine Garantien \"uber Threadsicherheit von stdin und stdout gibt.
UCIBridge erreicht dies, indem es als threadsicherer Singleton implementiert ist.
\pagebreak