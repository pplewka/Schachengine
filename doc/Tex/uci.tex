\def \citeuci[#1]{\cite[l. #1]{uci}}
\section{UCI}\label{sec:uci}
\subsection{Allgemein}\label{subsec:allgemein}
UCI ist eine standardisierte Schnittstelle zwischen grafischer Oberfl\"ache und Schachengine.
Die Kommunikation findet \"uber die Standardein- und Ausgabe (ab sofort stdin und stdout) der beiden Programme ab.\citeuci[9]
\newline
Die Engine soll zu jedem Zeitpunkt Befehle entgegennehmen und verarbeiten k\"onnen.\citeuci[15]
Deshalb ist in Baguette ein eigener Thread, genannt UCI-Thread, f\"ur die Kommunikation zust\"andig.
\newline
Den Programmablauf kann man in zwei Phasen einteilen:
\begin{enumerate}
    \item Phase: Initialisierung
    \item Phase: Schachspielen
\end{enumerate}
\subsubsection{Unabh\"angige Befehle}
Unabh\"angig von den Phasen sind nur die Befehle \textit{quit}, \textit{debug on}, \textit{debug off} und \textit{isready}.
\newline\textit{quit} beendet Baguette ordnungsgem\"ass.
Das Beenden auf andere Art (zum Beispiel Signale) kann zu korrupten Logs f\"uhren.
\newline \textit{debug [on off]} schaltet den Debugmodus ein bzw.\  aus.
Im Debugmodus werden interne Meldungen zur Fehlersuche mithilfe des \textit{info string} Befehls ausgegeben.
\newline \textit{isready} muss von der Engine mit \textit{readyok} beantwortet werden, um Bereitschaft zu signalisieren.
\subsubsection{Phase 1 Initialisierung}
Zuerst best\"atigen sich Engine und GUI, dass UCI verwendet wird mit \textit{uci} und \textit{uciok}. \citeuci[59 - 66]
Dann gibt die Engine ihren Namen und ihre Authoren mit \textit{id name} und \textit{id author} an.
Die Engine gibt der GUI mithilfe des \textit{option} Befehl alle m\"oglichen Optionen an.
Die GUI stellt Optionen mit dem \textit{setoption} Befehl ein und best\"atigt diese mit dem \textit{isready} Befehl.
Nur zu diesem Zeitpunkt erlaubt Baguette das \"Andern von Optionen entgegen der UCI Spezifikationen\citeuci[89], da
das Pflichtenheft verlangt, dass Optionen vor Spielstart eingestellt werden und manche Optionen zu komplex sind, um sie
w\"ahrend des Spiels zu \"andern (z.B.\ die Logdatei).
\newline \newline
Diese Phase wird nicht im UCI-Thread sondern im Controller-Thread ausgef\"uhrt.
Der Controller startet die \textit{initialize} Methode der UCI-Klasse.
Die UCI-Klasse liest die \textit{ucioption.properties} Datei aus und gibt ihren Inhalt an die \textit{initialize} Methode
der UCIBridge weiter.
Die UCIBridge initialisiert nun die Verbindung zu GUI bis zu den \textit{option} Befehlen.
Die UCIBridge gibt den Inhalt der \textit{ucioptions.properties} Datei an den UCIOptionHandler weiter.
Der UCIOptionHandler erstellt aus dem Inhalt der Datei die \textit{option} Befehle, sendet sie via UCIBridge an die GUI,
wertet die \textit{setoption} Befehle aus und gibt alle Optionen mit ihren Werten an die UCIBridge zur\"uck.
Die UCIBridge gibt die Optionen \"uber die UCI-Klasse an den Controller zur\"uck.
Die GUI ist jetzt bereit zu Spielen und der Controller kann abh\"angig von den Optionen den Rest von Baguette starten.
\subsubsection{Phase 2 Schachspielen}
Der UCI-Thread wartet auf eine Eingabe von der GUI\@.
Zuerst filtert die UCIBridge alle phasenunabh\"angigen Befehle aus den Eingaben aus und bearbeitet diese.
Die restlichen leitet sie weiter an die UCI-Klasse, nachdem sie alle unn\"otigen Leerzeichen entfernt hat.
Dies vereinfacht das weitere Auslesen der Eingaben.
Die UCI-Klasse verwirft alle Eingaben, ausser die, die oberfl\"achlich wie die Befehle \textit{ucinewgame}, \textit{position},
\textit{go} oder \textit{stop} wirken.
\subsubsection{UCI}
Die UCI-Klasse verbindet die UCIBridge mit dem Controller.
Da zu jedem Zeitpunkt nur ein UCI-Thread existieren soll, ist die UCI-Klasse als threadsicherer Singleton implementiert.
Der UCI-Thread startet die awaitCommandsForever Methode der UCI-Klasse.

\subsubsection{UCIBridge}
UCIBridge bildet die Verbindung zur grafischen Oberfl\"ache.
Keine andere Klasse hat Zugriff auf stdin und stdout, da Java keine Garantien \"uber Threadsicherheit von stdin und stdout gibt.
UCIBridge erreicht dies, indem es als threadsicherer Singleton implementiert ist.
\pagebreak