\section{Controller}\label{sec:controller}
\subsection{Allgemein}\label{subsec:allgemeincontroller}
Der Controller dient als Verbindungsst\"uck zwischen UCI, Timemanagement und der Suche.
Er wird vom Benutzer oder der GUI gestartet, startet die anderen Teile der Engine und vermittelt zwischen ihnen.
Der Controller l\"auft als eigenst\"andiger Thread, dem Controller-Thread.
Der Controller ist in der Klasse Controller, die das UCIListener-Interface erf\"ullt, implementiert.
\subsection{Start}\label{subsec:start}
Der Controller startet zuerst die UCI-Klasse;
diese startet selbst\"andig die restlichen Klassen zust\"andig f\"ur UCI\@.
Der Controller erh\"alt von der UCI-Klasse die eingestellten Optionen.
Abh\"angig von den Optionen erstellt der Controller den Logger.
Der Controller meldet sich als Beobachter bei der UCI-Klasse an.
Abh\"angig von der Prozessorkernanzahl startet der Controller eine Menge von Search-Threads.
Diese warten vorerst.
Dann startet der Controller den UCI-Thread.
\subsection{Spielvorgang}\label{subsec:controllerspielvorgang}
Der Controller wartet auf Befehle in seiner Befehlswarteschlange \textit{commandQueue}.
Die Befehle stammen von dem UCI- oder dem TimeManagement-Thread.
Der UCI-Thread legt potentiell die Befehle \textit{ucinewgame}, \textit{stop}, \textit{position} und \textit{go} in
\textit{commandQueue}.
Der TimeManagement-Thread legt potentiell den Befehl \textit{stop} in \textit{commandQueue}.
\newline
